\documentclass[]{article}

\usepackage{hyperref}
\usepackage[utf8]{inputenc}

\title{Creating a Voxel Based Game}
\author{Felix Bärring}

\begin{document}

\maketitle

\begin{abstract}
	
The purpose of this project is to create voxel based game, similar to games such as Infiniminer. The most characteristic feature is the ability to modify the world by removing and adding cubes. The player can move around in a seemingly infinite world, exploring and creating content. To make this possible, components such as rendering, collision detection, and procedural terrain generation must be implemented. Other components that can be relevant are, artificial intelligence and multi-player.

\end{abstract}

\section{Motivation and Background}

My Bachelor Thesis (available here {\url{https://gupea.ub.gu.se/handle/2077/39606}) was about creating a voxel game engine in Java using OpengGL, together with five other students. Although we were successful with reaching our goals, there were several aspects of the implementation that I think could have been done better. As I am interested in learning C++, I figured that doing a similar project in C++ instead of Java would be a good way to learn the language and various other topics, such as computer graphics and OpenGL.

\section{Goals}

My main goal with this project will be to learn as much as possible, if the game happens to be enjoyable that would of course be good, but it is not the main concern. The primary focus will be on rendering and world interaction. When those areas have been implemented other aspects such as Terrain Generation, AI, Audio, and GUIs will hopefully be implemented. I intend to make demos that showcase all the implemented functionality in isolation from other features, and also write short documentation on the subject.

\end{document}
